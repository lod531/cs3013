\documentclass[a4paper, 12pt]{article}
\usepackage{graphicx}
\graphicspath{ {images/} }
\usepackage[super]{nth}
\usepackage{booktabs}
\usepackage{enumitem}

\begin{document}
	\title{Own Idea - Computer Science Forum - Project Plan}
	\author{Group No. 2 - Screw you Gary, we'll call it what we want.}
	\maketitle
	\section{Project goals and objectives}
		\subsection{Background}
			\par The project was our own idea to do, hence doesn’t have the same context as other 
			projects. We are aiming to create an online forum for computer science students 
			of Trinity College Dublin to contribute to and benefit from, hence 
			we are treating our course-mates as our clients, and will ensure that they are happy with the final product. 

			\par Catering for computer science students, we will have to keep their various activities in mind, and
			 build a forum that suits their needs in regards to these activities. 
			For example, we need to be able to segment the forum into different 
			groupings based on year group, so that students can easily seek out 
			their peers. We also need to keep subject lists in mind, and 
			create sub-forums for each so that again students can seek out exactly what they need as soon as possible.
		\subsection{Objectives}
			\par Successful completion of the project should lead to increased communication amongst computer 
			science students, and hopefully an increased understanding of their course, 
			assignments and subject matter. The hope is that students will use the forum 
			to aid each other and contribute to meaningful discussions about their real world studies.

		\newpage
		\subsection{Goals}
			\begin{enumerate}[label*=\arabic*.]
				\item A specialised online message board for computer science students of Trinity College Dublin to contribute questions, answers and discussions about their course.
				\item Separated sub-forums per year grouping, which is further broken down into sub-forums for each topic and module.
				\item Unique information for each user and post. For example, unique user names, unique post dates, etc.
				\item Moderator tools to maintain the forum, help prevent spam and edit currently existing posts.
				\item Best Response System, where the original poster of a topic can highlight a response that they thought 
				was best/gained the most knowledge from so that other users in a similar position can easily get the same benefits.
			\end{enumerate}
	\newpage		
	\section{Project scope}
		\subsection{Project deliverables}
			\begin{enumerate}[label*=\arabic*.]
				\item Source code for front-end (PHP/HTML).
				\item Source code for back-end (SQL).
				\item Full documentation of database (entity relationship, relational schema, functional dependency diagrams etc).
				\item All documentation required by the module (i.e. weekly meeting minutes, project plan etc.)
			\end{enumerate}
		\subsection{Project Boundaries}
			\subsubsection{In scope}
				\begin{enumerate}[label*=\arabic*.]
					\item Unique user names.
					\item Trinity e-mail exclusive sign-up.
					\item Password protection.
					\item Topic creation.
					\item Replies to topics	
				\end{enumerate}
			\subsubsection{Out of scope}
				The below items will be attempted, should time permit it.
				\begin{enumerate}[label*=\arabic*.]
					\item Text editing tools (will attempt to integrate these if possible).
					\item Github integration.
				\end{enumerate}

	\newpage
	\section{Project approach}
		\subsection{Initial schedule}
			\begin{tabular}{ p{30 mm} p{50 mm} p{10 mm} p{10 mm} p{15 mm} }
				\toprule
				Stage & Description & Start & Finish & Duration (in days) \\
				\midrule
				Team Formation & Met as a team, broadly outlined goals and tasks for the project ahead & \nth{19} Jan & \nth{25} Jan & 7 \\ 
				\midrule
				Research and self education & Researched and self taught languages and techniques required. & \nth{25} Jan & \nth{8} Feb & 14 \\
				\midrule
				Project Work & \nth{2} years developing front end, \nth{3} years working on the back end & \nth{8} Feb & \nth{4} April & 56 \\
				\midrule
				Reflection & Review project and note successes and failures for future use & \nth{4} April & \nth{11} April & 7 \\
				\bottomrule
			\end{tabular}
		\subsection{Milestones}
			\begin{enumerate}[label*=\arabic*.]
				\item First front end implementation - \nth{15} February.
				\item Back end specification - \nth{22} February.
				\item First back end implementation \nth{29} February.
				\item First working prototype - \nth {14} March. 
				\item Security - \nth{21} March.
				\item Polished product - \nth{4} April.
			\end{enumerate}
		\subsection{Gantt chart and work breakdown structure}
			\par The gantt chart can be found appended to this document, as including it here would involve
			scaling the image at a factor such that the original image would be incomprehensible
	\newpage
	\section{Project organisation}
		\subsection{Roles and responsibilities}
			\begin{tabular}{p{35 mm} p{35 mm} p{54 mm}}
				\toprule
					Role & Group member & Responsibilities \\
					\midrule
					Project leader/Back end & Andrius Buinovskij & Allocating tasks, back end programming. \\
					\midrule
					Second in command/Back end & Conor McKenna & Taking minutes, writing documentation, back end programming. \\
					\midrule
					Front end & Joseph Fitzpatrick & Programming front end \\
					\midrule
					Front end & Laura Young & Programming front end \\
					\midrule
					Front end & Liam Farrelly & Programming front end \\
					\midrule
					Front end & Jerico Alcaras & Programming front end \\
				\bottomrule
			\end{tabular}

		\subsection{Staffing chart}
			\includegraphics[width=\textwidth]{staffing.png}

	\newpage
	\section{Risk analysis}
		\subsection{Risk analysis}
			\begin{tabular}{p{30 mm} p{30 mm} p{30 mm} p{30 mm}}
				\toprule
					Risk element & Impact (1 to 5) & Likelihood (1 to 5) & Risk factor (I*L) \\
					\midrule
					SQL Injection & 5 & 4 & 20 \\
					\midrule
					Inappropriate content & 1 & 5 & 5 \\
				\bottomrule
			\end{tabular}

		\subsection{Risk mitigation}
			\subsubsection{SQL Injection}
				SQL Injections will be dealt with in the manner standard within industry standard practices such as input sanitization
				and seperation of user input and actual SQL commends. 
			\subsubsection{Inappropriate conent}
				Inappropriate content is sure to arise, although hopefully in a jovial manner. Mitigation is left up to moderators and the
				users themselves.

	\newpage
	\section{Project Controls}
		\subsection{Scope}
			\par Project leaders set out the basic scope of the project from the first week. This includes all the basic 
			features of a forum, the things to store in the SQL database and an overview of what the front end design would be 
			like. To keep from being too overly ambitious in our approach, we agreed we would finish what we set out to 
			achieve in the basics first, before adding more to the scope of the project. 

		\subsection{Quality}
			\par All quality factors will be regulated via simple candid evaluation. A forum is a standard and simple concept, and
			our implementation will strive to simply adhere to the standard already widely prevalent.
		\subsection{Schedule}
			\par We set moderately strict deadlines for this project. After meeting on a Monday, 
			we would delegate ourselves work to have done for next week. The deadlines were met so 
			far, however we would set them in a way that if they weren’t met we would still be ahead of 
			schedule enough that it would be easy to keep or progress on track for the final submission.

	\newpage
	\section{Communications}
		\subsection{Client communications}
			\begin{tabular}{p{35 mm} p{35 mm} p{54 mm}}
				\toprule
					Meetings & Date & Purpose \\
					\midrule
					Initial meeting & \nth{22} Jan & Discuss features students would want to implemented in a CS forum. \\
					\midrule
					Post reading week meeting & \nth{7} Mar & Check up on the desired features implementation. \\
					\midrule
					Final meeting & \nth{28} Mar & Final review of the website with the  \\	
				\bottomrule
			\end{tabular}
	\section{Appendix}
		\subsection{Content}
			\begin{enumerate}[label*=\arabic*.]
				\item Requirements document.
				\item Change log (void, no changes have taken place).
				\item Compiled minutes.
			\end{enumerate}

\end{document}
