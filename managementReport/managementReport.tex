\documentclass[a4paper, 12pt]{article}
\usepackage{enumitem}

\begin{document}
\title{Management report}
\author{Group No. 2}
\maketitle

	\section{The project planning process}
		\par All of the projet planning was done via the weekly meetings, which were found to be sufficient. 
		Each week we would review our goals and set intermediate objectives. The team members would then choose
		out of the pool of available tasks, concluding the meeting. It's an extremely simple but a practical scheme, echoing agile development.
		\par The weekly meetings were essentially perfect for the scope of our project. A week's time was sufficient
		for implementation and development of most features and foundations, whilst being occasional enough not to
		bog down the team with excessive micro-management and unnecessary meetings.
		\par Ultimately, the project simply did not require any more than a week's foresight, since each week we
		did what we could to further development, and an overall plan was not going to change that metric.

	\section{Project goals and objectives}
		\par Since our project was an idea of our own, we were careful not to attempt to tackle anything too large.
	       	As a result, our goals were appropriately sized for our team. On a similar note, the idea, a computer science forum.
		being our own, was very well defined, and so no changes took place. We did precisely what we set out to do.

	\newpage
	\section{Project scope}
		\par Due to our own experience of cs3013 back in second year, we were very well aware of scope creep. To combat it, as mentioned
		before, we focused on developing a no bells-and-whistles forum, restricting ourselves to implementing core functionality and
		taking the time to implement it well, as opposed to rushing it in order to proceed to more interesting and exciting features. This
		served us well, as we accomplished our goals and built the fundamentals within the time frame we had hoped for, and furthermore, built
		those fundamentals well.

	\section{Project approach}
		\par As mentioned before, no tools were as useful as simple communication. Our project approach was simple, we were in it together,
		and did what we could. Since everyone was doing their best, there was no need for greater management, because there is only so
		much a person can do in a given time period. No amount of planning or discussion will increase that metric.
		\par Each week we would discuss what needed doing next, for example, forum registration needed to be complete. Within the meeting we
		would then discuss whether the goal, in this example, the forum registration, was too large to accomplish within a single week. If we decided
		it was manageable, then that was the end of it and we'd have a target for the week, splitting up the workload between the members (i.e.
		 one person would do e-mail verification, another interfacing with sql back-end etc.). If we decided that the goal was too large, it would
		be split, and the process would be repeated. 
	\section{Project organization}
		\par Our simple weekly-meetings only approach worked wonderfully for our small-scale project. Although it is not particularly scalable,
		it was quite appropriate for us. Each week goals would be set, those goals would always be completed come next week, and the cycle
		would be repeated. Honestly, it is a testament to the team in question, as opposed to the organization methodology. The combination of
		a small dedicated team and little-to-no planning overhead or micromanagement resulted in swift, pleasant and stress-free development.
	\section{Risk analysis}
		\par Our core risk, which was identified at the very start of the project, was user malice. Computer science students like to attempt
		to pick things apart for a variety of reasons, and our forum was sure to be a target. The main form of attack we had to protect against
		were SQL injections. Thankfully, seeing as our development was smooth and swift, we had leftover time to patch security issues and 
		sanitise user input to prevent such injections. No other risks were identified or came up, seeing as the project was a rather simple
		one.
	\section{Project controls}
		\subsection{Scope}
			\par As mentioned in the Project Scope section, scope creep was not an issue due to clearly defined goals at the beginning of
			the development. 
		\subsection{Schedule}
			\par Schedule was controlled via the weekly meetings. Seeing as the largest period of time between updates was a single week,
			no issues could grow to truly stall development. Thankfully, there was little to no actual controlling done, as every member
			"pulled their weight" without additional incentive to do so.
		\subsection{Quality}
			\par Quality was ensured via peer code-reviews, which is to say people looked at each other's code whilst developing. We have
			not had any issues with this form of quality control, but then again, the project is small-scale, so it is harder to make 
			big mistakes. However, as mentioned, it was sufficient for us.
	\section{Communications}
		\par The idea was our own, and so in lieu of a client, we spoke with our classmates, who are after all the target audience. It was
		effortless for us to ask for feedback, as we all had connections within the computer science course. The communication however
		yielded very little, seeing as the team, too, was the target audience, and between all of the team members we had a very good
		idea of what a computer science forum should be able to do.

\end{document}
